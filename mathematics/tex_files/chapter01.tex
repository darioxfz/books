%%% Local Variables: 
%%% mode: latex
%%% TeX-master: "../master.tex"
%%% End: 

\chapter{Algebra}
%%%%%%%%%%%%%%%%%%%%%%%%%%%%%%%%%%%%%%%%%%%%%%%%
\section{Why Study Algebra ?}
Algebra is the unifying thread of all mathematics, the language of the universe, the language of science.
It's an abstract system, meaning that you can describe really complex events in a more simple way thanks to the abstraction.


\section{Sets and Functions}


\paragraph{Definition.}

\textit{set} is a collection of objects.  We call the objects, 
\index{set}
\textit{elements}.   A set is denoted by listing the elements between
\index{elements!of a set}

$S \cup \emptyset = S$, $S \cap \emptyset = \emptyset$,
$S \setminus \emptyset = S$, $S \setminus S = \emptyset$. 

\paragraph{}
In general, we must take care in applying functions to equations.
If we apply a many-to-one function, we may introduce spurious
solutions.  Applying $f(x) = x^2$ to the equation
$x = \frac{\pi}{2}$ results in $x^2 = \frac{\pi^2}{4}$, which has the two solutions,
$x = \{ \pm \frac{\pi}{2} \}$.
Applying $f(x) = \sin x$ results in $x^2 = \frac{\pi^2}{4}$, which has 
an infinite number of solutions,
$x = \{ \frac{\pi}{2} + 2 n \pi \,|\, n \in \mathbb{Z} \}$.

