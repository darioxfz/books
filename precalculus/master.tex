 \documentclass[a4paper,11pt]{book}

%font
\usepackage[T1]{fontenc}
\usepackage[tracking]{microtype}
\usepackage[sc,osf]{mathpazo}   % With old-style figures and real smallcaps.
\linespread{1.025}              % Palatino leads a little more leading


%margins
\usepackage{geometry}
\geometry{
	a4paper,
	total={210mm,297mm},
	left=15mm,
	right=15mm,
	top=2mm,
	bottom=2mm,
}
%proofs
\usepackage{amssymb}
\newcommand{\justif}[2]{&{#1}&\text{#2}}
%
\usepackage{amsmath}
%math on tables
\usepackage{tabu}
%chapter number title position
\usepackage{quotchap}
%glossary



\author{Dario L}
\title{Precalculus}
\date{\today}

\begin{document}

\frontmatter
\maketitle

\chapter*{Preface}
The goal of this notes, is to have a formal reference of the important subjects necessary to understand calculus and more advanced subjects.
\\
\\
\textbf{Mathematics possesses not only truth, but supreme beauty, a beauty cold and austere, like that of a sculpture,
and capable of stern perfection, such as only great art can show.\\ --Bertrand Russell}



\tableofcontents

\mainmatter
%preface

%%%% Local Variables: 
%%% mode: latex
%%% TeX-master: "../master.tex"
%%% End: 

\chapter{Proofs}

\paragraph{What is a proof?} A proof is a method for ascertaining the truth.\\
\begin{itemize}
	\item Experimentation and Observation
	\item Sampling and counter-examples
	\item Judge, jury
	\item Word of god(religion)
	\item Word of your boss, or a client.
	\item The professor.
	\item Inner Conviction.
	\item I don't see why not.
\end{itemize}

\paragraph{What is a mathematical proof?} A mathematical proof is a verification of a set of propositions by a chain of logical deductions from a set of axioms.

\paragraph{Def.} A preposition is a statement that is either true or false.\\
ex.  $2+3=5$\\
ex. $\forall n \in \mathbb{N}, n^2+n+41$ is a prime number

\paragraph{Predicate} in the above case is ($n^2+n+41$ is a prime number). The predicate is a proposition whose truth depends on the value of variable(s).

20.00
%-----------------------------------------------------------------------------------------------------------------------------------------------------------------------------------------------------------------------------------
\chapter{Foundations}
\section{Greek Alphabet}

\centering
$\begin{tabu}{lllllll}
\cline{1-3} \cline{5-7}
\multicolumn{1}{|l|}{Letter} & \multicolumn{1}{l|}{Lower} & \multicolumn{1}{l|}{Upper} & \multicolumn{1}{l|}{} & \multicolumn{1}{l|}{Letter} & \multicolumn{1}{l|}{Lower} & \multicolumn{1}{l|}{Upper} \\ \cline{1-3} \cline{5-7} 
alpha & \alpha & A &  & nu & \nu & N \\
beta & \beta & B &  & xi & \xi & \Xi \\
gamma & \gamma & \Gamma &  & omicron & o & O \\
delta & \delta & \Delta &  & pi & \pi & \Pi \\
epsilon & \epsilon & E &  & rho & \rho & P \\
zeta & \zeta & Z &  & sigma & \sigma & \Sigma \\
eta & \eta & H &  & tau & \tau & T \\
theta & \theta & \Theta &  & upsilon & \upsilon & \Upsilon \\
iota & \iota & I &  & phi & \phi & \Phi \\
kappa & \kappa & K &  & chi & \chi & X \\
lambda & \lambda & \Lambda &  & psi & \psi & \Psi \\
mu & \mu & Mu &  & omega & \omega & \Omega
\end{tabu}$

\raggedright
\section{Language of mathematics}

The language of mathematics is a system to describe concrete ideas.
\section{Sets}
A set is a collection of distinct objects.
\subsection{Special Sets}
There are some sets that hold a great mathematical importance and are used regularly everywhere so they have acquire their own names and their conventions.

The empty set is one example, is usually denoted by $\emptyset$ or {}.
Different families of numbers have their own names as well like:
\begin{itemize}
	\item Prime Numbers - $\mathbb{P}$ or \textbf{P} = \{2, 3, 5, 7, 11, 13, 17, 19, 23, ...\}
	\item Natural Numbers - $\mathbb{N}$ or \textbf{N} = \{1, 2, 3, 4, ...\} sometimes 0 is considered as well
	\item Integers - $\mathbb{Z}$ or \textbf{Z} = \{..., -2, -1, 0, 1, 2, ...\}
	\item Real - $\mathbb{R}$ or \textbf{R} = Every number that can be found on the number line
	\item Complex Numbers - $\mathbb{C}$ or \textbf{C} = Every number that can be expressed in the form $a+bi$
	\item Irrational Numbers - $\mathbb{I}$ or \textbf{I} = any real number that cannot be expressed as a/b where a,b are integers
	\item Rational Numbers - $\mathbb{Q}$ or \textbf{Q} = any number that can be expressed as a/b where a,b are integers
\end{itemize}
\subsection{Operations}
There are several operations for construction new sets.
\subsubsection{Unions}
The union of \textbf{A} and \textbf{B} is denoted by $\textbf{A} \cup \textbf{B}$, can be also seen as the set of elements that belong to \textbf{A} or \textbf{B}.

\subsubsection{Intersections}
The intersection of \textbf{A} and \textbf{B} is denoted by $\textbf{A} \cap \textbf{B}$, can be also seen as the set of elements that belong to \textbf{A} and \textbf{B}. If \textbf{A} and \textbf{B} don't have any elements in common their intersection is the $\emptyset$ and they are said to be disjoint. 

\subsubsection{Complements}
A set complement is everything else that does not belong in it. $\textbf{A} \cap \textbf{A} = \Omega$


%-----------------------------------------------------------------------------------------------------------------------------------------------------------------------------------------------------------------------------------
\chapter{Functions}
A function $f$ is a rule that assigns to each element x in a set A exactly one element, called $f(x)$, in a set B

\textbf{MACHINEDIAGRAMHERE}
\subsubsection{A function can be represented in four different ways :}
\begin{itemize}
	\item verbally
	\item algebraically
	\item visually
	\item numerically
\end{itemize}
\section{Linear}
A linear equation in one variable is an equation equivalent to one of the form $$ ax+b=0 $$ where a and b are real numbers and x is the variable.
%-----------------------------------------------------------------------------------------------------------------------------------------------------------------------------------------------------------------------------------

\section{Quadratic}
A quadratic equation is and equation of the form $$ax^2+bx+c=0 $$ where a,b, and c are real numbers with a $\neq$ 0
\subsubsection{zero-product property}
$AB=0 \iff A=0$ or $B=0$
\subsubsection{factoring}
\subsubsection{Completing the square}
\subsubsection{Quadratic Formula}
The roots of the quadratic equation $ax^2+bx+c=0$, where $a\neq0$ are $$ x = \frac{-b\pm\sqrt{b^2-4ac}}{2a}$$
\textbf{PROOF:}
\begin{alignat*}{2}
					0 &= ax^2+bx+c                                             \justif{\quad}{starting from the base polynomial}\\
	   \frac{-c}{a} &= x^2+\frac{b}{a}x                                        \justif{\quad}{divide by a}\\
    x^2+\frac{b}{a}x+(\frac{b}{2a})^2&= -\frac{c}{a}+ (\frac{b}{2a})^2         \justif{\quad}{completing the square with $(\frac{b}{2a}$})^2 \\
    (x+\frac{b}{2a})^2&=\frac{-4ac+b^2}{4a^2}                                                  \justif{\quad}{Perfect Square}\\
	 x+\frac{b}{2a}  &= \pm\frac{\sqrt{b^2-4ac}}{2a}                          \justif{\quad}{Take square root}\\
    x &= \frac{-b\pm\sqrt{b^2-4ac}}{2a}   \justif{\quad}{subtract $\frac{b}{2a}$}  
\end{alignat*}

\textbf{Descriminant:} the discriminant of the general quadratic $ax^2+bx+c=0 (a\neq0) $ is $D=b^2-4ac$ where
\begin{itemize}
\item if $D > 0$ then the equation has two distinct real solutions.
\item if $D = 0 $ then the equation has exactly one real solution.
\item if $D < 0$ then the equation has no real solution.
\end{itemize}
\section{Rational and Polynomial}
\section{Exponential and Logarithmic}
\subsection{Binomial Theorem}
\section{Trigonometric}
\section{Hyperbolics}
%-----------------------------------------------------------------------------------------------------------------------------------------------------------------------------------------------------------------------------------
\chapter{Inequalities}
\chapter{Lines}
The slope of a line or the steepness of a line is how quicly it rises or falls as we move from right to left. We define run to be the distance in the x axis and rise to be the corresponding distance in the y axis. The slope of a line can be then be expressed by
$$ slope = rate of change = m = \frac{rise}{run} = \frac{y}{x} = \frac{y_2-y_1}{x_2-x_1}$$

\subsubsection{General Equation of a line}
The graph of every \textbf{linear equation} $Ax+By+C=0$ if (A,B not both zero) is a line. 
\subsubsection{point-slope form of the equation of a line}
$$ y-y_1=m(x-x_1)$$
\subsubsection{slope-intercept form of the equation of a line}
$$y=mx+b$$
\subsubsection{parallel lines}
two nonvertical lines are parallel $\iff$ they have the same slope
\subsubsection{Perpendicular lines}
two lines with slopes $m_1$ and $m_2$ are perpendicular $\iff m_1m_2=-1$, that is their slopes are negative reciprocals: $$ m_2 = -\frac{1}{m_1}$$ also, a horizontal line $m=0$ is perpendicular to a vertical line.
\chapter{Geometry}
\chapter{Sequences and Series}
\chapter{Conics}
\chapter{Trigonometry}
%-----------------------------------------------------------------------------------------------------------------------------------------------------------------------------------------------------------------------------------
\end{document}
