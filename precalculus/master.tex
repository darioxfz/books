 \documentclass[a4paper,11pt]{book}

%font
\usepackage[T1]{fontenc}
\usepackage[tracking]{microtype}
\usepackage[sc,osf]{mathpazo}   % With old-style figures and real smallcaps.
\linespread{1.025}              % Palatino leads a little more leading


%margins
\usepackage{geometry}
\geometry{
	a4paper,
	total={210mm,297mm},
	left=15mm,
	right=15mm,
	top=2mm,
	bottom=2mm,
}

%
\usepackage{amsmath}
%math on tables
\usepackage{tabu}
%chapter number title position
\usepackage{quotchap}
%glossary


\author{Dario L}
\title{Precalculus}
\date{\today}

\begin{document}

\frontmatter
\maketitle

\chapter*{Preface}
The goal of this notes, is to have a formal reference of the important subjects necessary to understand calculus and more advanced subjects.
\\
\\
\textbf{Mathematics possesses not only truth, but supreme beauty, a beauty cold and austere, like that of a sculpture,
and capable of stern perfection, such as only great art can show.\\ --Bertrand Russell}



\tableofcontents

\mainmatter
%preface

%%%% Local Variables: 
%%% mode: latex
%%% TeX-master: "../master.tex"
%%% End: 

\chapter{Algebra}
%%%%%%%%%%%%%%%%%%%%%%%%%%%%%%%%%%%%%%%%%%%%%%%%
\section{Why Study Algebra ?}
Algebra is the unifying thread of all mathematics, the language of the universe, the language of science.
It's an abstract system, meaning that you can describe really complex events in a more simple way thanks to the abstraction.


\section{Sets and Functions}


\paragraph{Definition.}

\textit{set} is a collection of objects.  We call the objects, 
\index{set}
\textit{elements}.   A set is denoted by listing the elements between
\index{elements!of a set}

$S \cup \emptyset = S$, $S \cap \emptyset = \emptyset$,
$S \setminus \emptyset = S$, $S \setminus S = \emptyset$. 

\paragraph{}
In general, we must take care in applying functions to equations.
If we apply a many-to-one function, we may introduce spurious
solutions.  Applying $f(x) = x^2$ to the equation
$x = \frac{\pi}{2}$ results in $x^2 = \frac{\pi^2}{4}$, which has the two solutions,
$x = \{ \pm \frac{\pi}{2} \}$.
Applying $f(x) = \sin x$ results in $x^2 = \frac{\pi^2}{4}$, which has 
an infinite number of solutions,
$x = \{ \frac{\pi}{2} + 2 n \pi \,|\, n \in \mathbb{Z} \}$.


%-----------------------------------------------------------------------------------------------------------------------------------------------------------------------------------------------------------------------------------
\chapter{Foundations}
\section{Greek Alphabet}

\centering
$\begin{tabu}{lllllll}
\cline{1-3} \cline{5-7}
\multicolumn{1}{|l|}{Letter} & \multicolumn{1}{l|}{Lower} & \multicolumn{1}{l|}{Upper} & \multicolumn{1}{l|}{} & \multicolumn{1}{l|}{Letter} & \multicolumn{1}{l|}{Lower} & \multicolumn{1}{l|}{Upper} \\ \cline{1-3} \cline{5-7} 
alpha & \alpha & A &  & nu & \nu & N \\
beta & \beta & B &  & xi & \xi & \Xi \\
gamma & \gamma & \Gamma &  & omicron & o & O \\
delta & \delta & \Delta &  & pi & \pi & \Pi \\
epsilon & \epsilon & E &  & rho & \rho & P \\
zeta & \zeta & Z &  & sigma & \sigma & \Sigma \\
eta & \eta & H &  & tau & \tau & T \\
theta & \theta & \Theta &  & upsilon & \upsilon & \Upsilon \\
iota & \iota & I &  & phi & \phi & \Phi \\
kappa & \kappa & K &  & chi & \chi & X \\
lambda & \lambda & \Lambda &  & psi & \psi & \Psi \\
mu & \mu & Mu &  & omega & \omega & \Omega
\end{tabu}$

\raggedright
\section{Language of mathematics}

The language of mathematics is a system to describe concrete ideas.
\section{Sets}
A set is a collection of distinct objects.
\subsection{Special Sets}
There are some sets that hold a great mathematical importance and are used regularly everywhere so they have acquire their own names and their conventions.

The empty set is one example, is usually denoted by $\emptyset$ or {}.
Different families of numbers have their own names as well like:
\begin{itemize}
	\item Prime Numbers - $\mathbb{P}$ or \textbf{P} = \{2, 3, 5, 7, 11, 13, 17, 19, 23, ...\}
	\item Natural Numbers - $\mathbb{N}$ or \textbf{N} = \{1, 2, 3, 4, ...\} sometimes 0 is considered as well
	\item Integers - $\mathbb{Z}$ or \textbf{Z} = \{..., -2, -1, 0, 1, 2, ...\}
	\item Real - $\mathbb{R}$ or \textbf{R} = Every number that can be found on the number line
	\item Complex Numbers - $\mathbb{C}$ or \textbf{C} = Every number that can be expressed in the form $a+bi$
	\item Irrational Numbers - $\mathbb{I}$ or \textbf{I} = any real number that cannot be expressed as a/b where a,b are integers
	\item Rational Numbers - $\mathbb{Q}$ or \textbf{Q} = any number that can be expressed as a/b where a,b are integers
\end{itemize}
\subsection{Operations}
There are several operations for construction new sets.
\subsubsection{Unions}
The union of \textbf{A} and \textbf{B} is denoted by $\textbf{A} \cup \textbf{B}$, can be also seen as the set of elements that belong to \textbf{A} or \textbf{B}.

\subsubsection{Intersections}
The intersection of \textbf{A} and \textbf{B} is denoted by $\textbf{A} \cap \textbf{B}$, can be also seen as the set of elements that belong to \textbf{A} and \textbf{B}. If \textbf{A} and \textbf{B} don't have any elements in common their intersection is the $\emptyset$ and they are said to be disjoint. 

\subsubsection{Complements}
A set complement is everything else that does not belong in it. $\textbf{A} \cap \textbf{A} = \Omega$


%-----------------------------------------------------------------------------------------------------------------------------------------------------------------------------------------------------------------------------------
\chapter{Functions}
\section{Linear}
%-----------------------------------------------------------------------------------------------------------------------------------------------------------------------------------------------------------------------------------

\section{Quadratic}
\section{Rational and Polynomial}
\section{Exponential and Logarithmic}
\subsection{Binomial Theorem}
\section{Trigonometric}
\section{Hyperbolics}
%-----------------------------------------------------------------------------------------------------------------------------------------------------------------------------------------------------------------------------------
\chapter{Inequalities}
\chapter{Geometry}
\chapter{Sequences and Series}
\chapter{Conics}
\chapter{Trigonometry}
%-----------------------------------------------------------------------------------------------------------------------------------------------------------------------------------------------------------------------------------
\end{document}
